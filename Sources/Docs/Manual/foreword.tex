\chapter*{Введение}
\addcontentsline{toc}{chapter}{Введение}
\chaptermark{Введение}

Фреймворк «ManagedIrbis» предназначен для организации программного доступа к ресурсам, находящихся под управлением сервера ИРБИС64, и может использоваться для как для расширения функциональности стандартных АРМ, входящих в поставку АБИС ИРБИС64, так и для создания собственных программных продуктов, совместимых с ИРБИС64.

\section*{Совместимость}

Фреймворк совместим со следующими версиями ИРБИС64:
2004	2005	2006	2007	2008	2009	2010	2011	2012	2013	2014	2015

Совместимость с конкретной версией сервера ИРБИС64 устанавливается по результатам прогона набора стандартных тестов: подключение к серверу, получение служебной информации (версия сервера, количество лицензий и т. д.), чтение записей, форматирование записей, сохранение записей и т. д.

\section*{Инструментарий}

Фреймворк написан на языке C\# в среде Microsoft Visual Studio 2013 для Micorsoft .NET Framework 4.5. Для сборки библиотеки из исходных текстов необходим совместимый инструментарий: Visual Studio 2013 или более новой версии как бесплатной редакции (Express), так и платной (Standard, Profes\-sional и т. д.).

Библиотека должна без модификации успешно собираться средами Mono\-Develop (версия не ниже 4.0) и Sharp\-Develop (версия не ниже 4.4).
Однако всё многообразие альтернативного инструментария не было протестировано авторами (и они не ставили перед собой подобной задачи), и авторы рекомендуют использовать для сборки Visual Studio 2013.

\section*{Системные требования}

Основным системным требованием библиотеки является наличие Microsoft.Net framework 4.5/4.5.1/4.5.2 или совместимой с ним среды исполнения управляемого кода.
Фреймворк должен функционировать в следующем окружении:

\begin{table}[htbp]
	\centering
	\caption{Поддерживаемые окружения}
	\begin{tabular}{ | p{0.4\textwidth} | p{0.4\textwidth} | }
	\hline
	\textbf{Окружение} & 
	\textbf{Функционирование, требования}
	\\
	\hline
	\hline
	Microsoft Windows XP & Не поддерживается \\
	\hline
	Microsoft Windows Vista SP2 & Необходимо установить .Net framework 4.5 \\
	\hline
	Microsoft Windows Server 2003 & Не поддерживается \\
	\hline
	Microsoft Windows 7 SP1 & Необходимо установить .Net framework 4.5 \\
	\hline
	Microsoft Windows Server 2008 SP2/2008 R2 SP1 & Необходимо установить .Net framework 4.5 \\
	\hline
	Microsoft Windows Server 2012/2012 R2 & Предустановлен в операционной системе \\
	\hline
	\end{tabular}
\end{table}

\section*{ManagedIrbis в Интернет}

Исходные коды фреймворка размещены на Git-хостинге github.com по адресу https://github.com/amironov73/arsmagna. Доступ к репозиторию открыт на чтение для всех.
Исполняемые файлы фреймворка опубликованы на сервисе NuGet по адресу

\section*{Лицензия}

Фреймворк распространяется как продукт с открытым исходным кодом. Любой желающий может:

\begin{itemize}
	\item Использовать бинарный релиз библиотеки в своих проектах в неизменном виде – в этом случае требуется лишь указание на авторство библиотеки.
	\item Адаптировать исходный код для собственных нужд и использовать в своих проектах модифицированную версию библиотеки или [модифицированные] фрагменты кода из неё – в этом случае требуется указание на авторство библиотеки и факт модификации её кода.
\end{itemize}

Никаких лицензионных отчислений в вышеперечисленных случаях не требуется. 

\section*{Благодарности}

Авторы выражают благодарность:

\begin{itemize}
	\item \textbf{Ивану Батраку} (СФУ), протестировавшему библиотеку на совместимость со старыми версиями ИРБИС-сервера;
	\item \textbf{Арсению Валентиновичу Шувалову} (Саратовская государственная консерватория им. Л. В. Собинова), выявившему ошибки в библиотеке;
	\item \textbf{Артёму Васильевичу Гончарову} (Научная музыкальная библиотека Санкт-Петер\-бургской Консерватории им. Н. А. Римского-Корсакова), выявившему некоторые досадные ошибки в библиотеке.
\end{itemize}

\section*{Версии и совместимость}

Данное руководство описывает версию 1.3.0.24 библиотеки. Версия библиотеки физически хранится как ресурс VERSION сборки ManagedClient.dll и как статическое свойство Version класса ManagedClient64.

Подробнее о проверке версий см. пункт «Определение версии сервера и клиента».
На данный момент несовместимых версий библиотеки нет, поэтому обновление может осуществляться простым копированием новой сборки поверх старой.
Все будущие несовместимости, если таковые появятся, будут описаны в данном разделе.


